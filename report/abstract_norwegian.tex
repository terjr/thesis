\pagestyle{empty}
\renewcommand{\abstractname}{Sammendrag}
\begin{abstract}

    \noindent Energieffektivitet er en av de største utfordringene i moderne
    datamaskindesign. Videre ytelsesøkning begrenses av høy strømtetthet, i
    tillegg har energieffektivitet stor betydning i alt fra strømregningen på
    superdatamaskiner til batterlevetid for små innebygde enheter. I denne
    masteroppgaven ser vi nærmere på arkitekturen og energiforbruket til en ARM
    Cortex-A9. Vi lager deretter et verktøy for å forutsi dens strømforbruk gjennom
    simulering.

    Via målinger og eksperimenter gjort på ekte maskinvare bestemmes
    strømforbruk på instruksjonsnivå. Videre blir dette koblet til
    bestemte hendelsesforløp i den samme arkitekturen modelert i gem5-simulatoren.
    Verktøyet vårt benytter så disse hendelsene, sammen med loggfiler fra
    simulatoren, til å lage en representasjon av prosessorens strømforbruk over
    tid.

    Vår metode kan benyttes i prosessorutvikling allerede i simulatorfasen, mens
    tradisjonelle metoder ikke virker før maskinvaren er ferdig syntetisert.
    Resultatene viser at verktøyet vårt kan estimere strømforbruk innenfor 10~\%
    feilmargin på normale arbeidslaster. Det kan også identifisere
    positive og negative utviklinger i strømforbruket gjennom kjøringen av et
    program.

\end{abstract}
