\pagestyle{empty}
\renewcommand{\abstractname}{Sammendrag}
\begin{abstract}
\noindent Energieffektivitet en av de største utfordringene i moderne
datamaskindesign. Videre ytelsesøkning begrenses av høy strømtetthet, i
tillegg har energieffektivitet stor betydning i alt fra strømregningen på
superdatamaskiner til batterlevetid for små innebygde enheter. I denne
masteroppgaven ser vi nærmere på arkitekturen og energiforbruket til en ARM
Cortex-A9 og lager et verktøy for å forutsi dens strømforbruk gjennom
simulering.

Via målinger og eksperimenter gjort på ekte hardware blir
instruksjonsnivå-strømforbruk bestemt. Videre blir dette koblet til
hendelsesforløp i den samme arkitekturen simulert i gem5.  Verktøyet benytter
disse hendelsene sammen med loggfiler fra simulatoren tol å lage en
representasjon av prosessorens strømforbruk over tid.

Vår metode kan benyttes i prosessorutvikling allerede i simulatorfasen 

\end{abstract}
