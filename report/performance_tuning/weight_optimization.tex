\section{Multi-objective Weight Optimization}

With an accurate CPU model, the weights must now be tuned to match measured
power consumption on real hardware. We formulate this as a multi-objective
optimization problem.

We start by creating a set of training workloads, $\{T_1 \dots T_n \}$.
Essentially, these are computer programs designed to hold certain
characteristics compiled to a native ARMv7 binary. The design and selection of
these will be elaborated in the next section. The binaries are being executed in
the simulator with the CPU model from last section and \texttt{(time,
event)} tuples are outputted. We must now decide on a cost for each of the event
types.

We attack this problem by running an optimization
algorithm that tests various combinations of the weight configuration. We tested
many evolutionary strategies and were able to prototype quickly using DEAP
\cite{DEAP_JMLR2012}, a Python framework for evolutionary algorithms. In the
end, we do not care how the weights are found, as long as they match well to the
hardware measurements. Please note that any optimization algorithm could be used
to find a proper set of weights.

The genome is a set of CPU events, each mapped to an energy cost. E.g.
\texttt{\{IntAlu: 170, IntMult: 1300, MemRead: 80, MemWrite: 50, SimdFloatMisc:
1400, L1R: 230, L1W: 340, L2R: 1100, L2W: 1300, PhysR: 2600, PhysW: 2800\}} is a
valid individual in the population. To calculate the fitness of an individual,
we run PET with the genome weights on a set of workloads and compare the energy
profiles with measurements on hardware. Note that hardware measurements only
needs to be done once per workload. The evolution can now run without
supervision until a set of reasonable weights are evaluated.
