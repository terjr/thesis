\section{Output}
Depending on application, different output formats might come in handy. PET
supports printing plain time power dump, a somewhat more readable table to
plain, table and a gnuplot figure with time as X-axis and power drain on the
Y-axis. The output format is understood as timeslots in which the architecture
has a certain current drain, which should be multiplied with applied voltage to
get numbers on consumed energy, heat generation and so on. The numbers are
scaled as milliamperes, which of course is equal to milliwatt if voltage is
$1V$.

All output formats supports annotation of function calls. This is achieved by
giving PET a list of symbols and their corresponding program counter value. PET
will tag each power bucket with the last seen symbol within that bucket, as this
symbol is the one most likly to continue over to the next measure point. Lists
of symbols can be extracted from non-stripped binaries using the script listed
in \autoref{lst:annotscript}. This script is also included in the \texttt{scripts}-folder
of PET.

\lstinputlisting[float,label={lst:annotscript},caption={Extract symbols from binary},language=sh]{../pet/scripts/annotate.sh}

Example of the \texttt{plain} output format can be seen in
\autoref{lst:pet_output_plain}. The left column is the bucket number, while the
right column is instant current draw from the modelled architecture.

\begin{lstlisting}[float,label={lst:pet_output_plain},caption={PET Plain Output}]
0 120 memcpy
1 113 start
2 150 main
3 123 main
4 133 fun1
5 117 main
\end{lstlisting}

When reading the output directly from console, a more descriptive output format
is the \texttt{table} format. An example using this option is rendered in
\autoref{lst:pet_output_table}.

\begin{lstlisting}[float,label={lst:pet_output_table},caption={PET Table Output}]
/----------------------------------------\
|   Bucket   |   Energy   |    Symbol    |
|------------|------------|--------------|
|          0 | 120.000000 |    memcpy    |
|          1 | 113.000000 |    start     |
|          2 | 150.000000 |    main      |
|          3 | 123.000000 |    main      |
|          4 | 133.000000 |    fun1      |
|          5 | 117.000000 |    main      |
\----------------------------------------/
\end{lstlisting}

Visualization is often a good thing when inspecting old or trying to understand
new problems. As it is hard to get a good overview from huge text log files, PET
provides, as stated in \autoref{subsec:annot}. In effect, it is formatting
temporary output files and calling GNUPlot do do the hard work.

\begin{figure}
    \includegraphics[width=0.9\textwidth]{figs/annot.pdf}
    \caption{PET Annotated Graphical Output}
    \label{fig:annot}
\end{figure}

