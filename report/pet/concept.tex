\section{Concept}
The goal for PET and this project is to ultimatly estimate power usage and thus
also energy efficiency of still not yet implemented computer architectures. A
such approach will of course reside well within the term "estimation" and will
doubtly reach the correct numbers. Never the less, PET is built by measuring
real hardware with great detail, capturing discrete events and assigning each
event a certain amount of energy consumption.

As depicted in \autoref{fig:workflow}, when selected events have been weighted,
one can run the test program though a simulator set up to act as the new
hardware.  The simulator will generate a tracelog containing the weighted
events, and PET can then apply the numbers. From this workflow, PET can produce
a data set containing power consumption distributed over the simulation
lifetime.  As noted, the new hardware will be weighted equally of a chosen
existing hardware, so so this method requires a certain similarity beteen the
new and the old hardware. In general, all traditional architectures contains
equal principles of function, and is thus mappable to each other, but accuracy
will of course differ as hardware differs in design, applied voltage, clock
speed and process technology.

\begin{figure}
    \includegraphics[width=0.9\textwidth]{figs/pet-workflow-gv.pdf}
    \caption{Workflow using PET}
    \label{fig:workflow}
\end{figure}

When estimating power usage for new architectures, it is hard to tell exactly
how to chose events and how expensive each event is. Therefore, we have to
assume that the new architecture is comparable to an old architecture, and thus
we can use the same power estimation numbers to give a new estimate.
