\section{Demand for Energy Efficiency}

We are now at the beginning of an era where energy efficiency and performance
are tightly coupled. When improving performance, one must take care not to
exceed the physical limitation of power dissipation. Thus, energy efficiency is
key to additional performance gain; performance per Watt must be emphasized.

Heat is not the only motivating factor to keep energy consumption down.
Processors targeting laptops, cellphones and other mobile devices have always
been energy-constrained due to their use of batteries. Lower energy consumption
would allow for longer battery life and/or heavier applications. More recently,
mobile processors have become increasingly popular in alternative domains,
such as supercomputing. Their low cost and high performance per Watt ratio makes
them attractive for massively parallel problems, which is currently done on
large and expensive supercomputers. These machines have huge energy budgets and
are taken out of service after just a couple of years, being replaced by new
machines that offer better performance for less power. Building data centers
from low-cost embedded processors is believed to have a huge potential and
could change the landscape of supercomputing in the future
\cite{rajovic2013supercomputing}.

Not only data centers benefit from the use of mobile processors. The SHMAC
research project at NTNU aims to build a single-ISA heterogeneous computing
platform with processing cores specialized for energy efficiency. Using the most
efficient processor or hardware accelerator -- in terms of both energy and
performance -- is the key to success for such platforms.

There are several reasons to minimize a processors energy consumption. Batteries
would last longer, applications become richer and it will enable processor
performance growth to continue. Energy efficiency has become crucial;
performance alone is no longer the single most important attribute of
processors.

