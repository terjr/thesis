\section{Hardware Simulators}

\subsection{A Brief Comparison}
\label{subsec:simulators}
There are numerous Computer Architecture Simulators available to the public,
both commercial, free and open source alternatives exists. In order to estimate
power consumption, the simulator needs to be able to give a good estimate of the
events which will happen in the hardware implementation of the architecture, as
this is the base for the calculations. The hardware platform used in this thesis
is an out-of-order ARM Cortex-A9 based core fitted with memory and and
peripherals in a von Neumann architectural scheme. The out-of-order propery
significantly increases the level of complexity which the simulator must handle.

\begin{description}
\item[Sniper] \hfill\\
    Sniper is a high-speed, multicore, multi-threaded and cycle-accurate
    computer architecture simulator \cite{sniperwebpage,carlson2013ssomta}. It
    already integrates with McPAT, and it is open source. Sniper does only work
    x86, and is thus not applicable for simulation of ARM-based architectures.

\item[SimpleScalar] \hfill\\
    SimpleScalar is a popular commercial architectural simulator that comes with
    a free academic license providing full source code. SimpleScalar supports
    the ARM instruction set amongst many others, and looks like a decent
    simulator for advanced out-of-order core simulation. SimpleScalar is also
    the simulator used by the Wattch-project \cite{brooks2000wattch}. However,
    the SimpleScalar web page seems to be last updated in 2004 and all mailing
    list archives are gone for the moment of writing. The source code for
    SimpleScalar v3 was still available and has been changed in 2011, but the
    entire SimpleScalar core has not changed since 2003.

\item[QEMU]\hfill\\
    QEMU is a generic and open machine emulator witch enables near real-time
    performance on architectures such as ARM, even when the host machine is
    running x86. However, QEMU is a machine emulator rather than an
    architectural simulator, thus despite it's great performance of running
    ARM-binaries, it will not produce CPU and memory event trace logs, and
    cannot be used in this project.

\item[gem5]\hfill\\
    gem5 is a merger \cite{gem5hipeac} between the M5 simulator
    \cite{binkert2006m5} project and the GEMS simulator project \cite{GEMS}.
    gem5 provides ARM-support with out-of-order execution out of the box, and
    provide great and cycle-accurate trace logs which is very applicable for
    this project \cite{gem5simulator}. gem5 is written in C and is configured
    with Python scripts, which makes it not so difficult to extend or change.
    Many of the maintainers are employees of ARM Corp., and the activity on the
    mailing lists suggests high project activity \cite{gem5dev}.
\end{description}

Provided this comparison of simulators, and given the fact that NTNU has a tradition
for using M5 and gem5 in earlier research projects and subjects, gem5 is the natural
winner for the choise of an architectural simulator.


\subsection{The gem5 Simulator}

Joint effort between many academic and industrial institutions has resulted in
an open source computer simulation framework, the gem5 simulator
\cite{gem5,gem5simulator}. Its core is written in C++ and has a highly modular
interface that allows users to specify simulator target through simple Python
scripts.

The project merges the best features of M5 \cite{binkert2006m5} and GEMS
\cite{GEMS} and includes a wide range of CPU and memory models
\cite{gem5hipeac}. The simulator has two main execution modes; \textit{Syscall
Emulation} (SE) or \textit{Full System} (FS). In SE mode, the simulator traps
system calls in the binary and emulates them, often by passing them to the host
operating system. In FS mode, the simulator can load an operating system binary
and run application within the OS. The latter mode is well suited when the OS in
its entirety must be simulated. gem5 supports many architectures; it can run
binaries compiled for ALPHA, SPARC, MIPS, ARM, x86 and POWER architectures.

During simulation, gem5 keeps track of hundreds of events related to the CPU
core and memory system. In-detail statistics, similar to performance counters on
real hardware, are then dumped for off-line inspection. gem5 can also output a
trace of what it is currently doing, originally intended for debugging of gem5.
These traces grow quickly in size, but provide useful insights of the simulated
execution. In particular, it describes CPU activity down to the
microarchitecture level and outputs simulated processor activity.

Drawing energy profiles from the trace files can be done by simply adding a cost
to the different types of CPU activity. Finding the weights with best fit
(as compared with measurements on real hardware) is simply the problem of
parameter optimization.
