\section{Hardware Platform}

The results contributed through this work relies on a method to isolate and
measure core voltage on a hardware implemented reference CPU. One such platform
where the core voltage is easily externally accessible is the ODROID-X2
Development Platform\cite{hardkernelodroidx2}. This board is equiped with a
Samsung Exynos 4412 ``Prime'', a modern System-on-Chip featuring four ARM
Cortex-A9 CPU cores, Mali-400 GPU and 2GB of on-chip DRAM. The Cortex-A9 is one
of ARM's mid-to-high-range application processors. It features an out-of-order
dual-issue speculative RISC core, and it is designed with energy efficiency in
mind. This processor is primarily found in low-powered embedded devices with
some performance demands, typically smartphones and tablets.

\begin{table}
\begin{tabular}{|c|c|}
\hline
Component   & Spesification\\
\hline
SoC         & Samsung Exynos 4412 ``Prime'' \\
CPU         & Quad-core ARM Cortex-A9 1.7GHz \\
Main memory & 2GB LP-DDR2 880MHz \\
L1          & Dual 32KB \\
L2          & 1MB \\
\hline
\end{tabular}
\caption{Hardware specifications ODROID-X2}
\label{tab:hwspecx2}
\end{table}

After the decode stage, an out-of-order multi-issue module with speculation can
schedule two arithmetic operations, in which one can be a multiply (the
processor has only one hardware multiply unit). It also has a multiplexed lane
for address operations and floating point operations (the \emph{NEON} FPU). The
Cortex-A9 is often used as 1-4 cores\cite{armsite}, where the implementation
used on the Exynos 4412 is a 4-core variant\cite{somesite}. In this experiment,
3 of the cores are disabled to ease both the measurement and the simulation
process.

For simulation correctness, good architectural knowledge is crucial, but most
proparitary architectures have a lot of details that remain undisclosed to the
public. Different sources have been used to find properties needed by the simulater,
some details regarding the OoO cores can be read about in \cite{blem2013detailed}.
