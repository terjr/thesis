\section{Conclusion}

% Hva
Throughout this thesis we have created and evaluated PET, a tool for estimating
power consumption on existing as well as non-existing computer architectures.
PET uses output from gem5 together with a set of weighted parameters to estimate
energy consumption of a program on a given hardware model. The weighted
parameters are selected by investigating the pipeline of an ARM Cortex-A9
processor. We have run a set of workloads on the hardware platform an logged
their current drain over time. Further, the results was used as input for a
genetic algorithm that mapped the correct energy usage to each architectural
event.

PET is not designed to be as accurate as possible, but to assist hardware
developers as early in the design stage as possible. As opposed to classical
methods, PET can be applied to a design already when only a simulation model
exists. Well known tools such as Wattch or McPAT also utilizes a simulator, but
requires a more knowledge about the final hardware, e.g. RTL and process
technology. Regardless, PET is able to estimate current drain within 10~\% of
actual drain when testing against the ARM Cortex-A9 processor.

Since PET is tool meant to be used early and rapidly in the design phase, it
has to be fast and easy to use. PET will predict power usage from log files,
reading of a speed of 133 MB/s on when run on an Intel Core i7 4820, and
needs only the log file and a set of weights to operate. Even when log files
easily expands to more than 10s of gigabytes, running PET is cheaper in terms of
time than running gem5.

an excessive amount of time has been used for tweaking the PET, gem5 and the genetic
algorithm to match the real hardware as good as possible. We believe that fitting
PET and its environment to a new processor technology 


  results as training data

These results was
used as the 
a genetic algorithm is used to map correct energy usage to 


  found by combining energy measurements on real hardware with a genetic
search algorithm to find 
















When hardware developers have access to tools like PET, the cost of doing energy
efficiency estimations will fall to a much lower level. It is not unusual to
have to go all the way to the RTL level and use some sort of low-level power
modeling such as SPICE \cite{ponomarev2002accupower} or PowerTheater
\cite{bruno2005rtl} to get values of decent accuracy. Wattch and McPAT has
provided some sort of easier access to architectural level power estimation, but
is still requires too much effort to be really easy in use. PET simply needs a
trace log from gem5, or with minor modifications, any application able to output
some sort of trace log. It seems that PET allows evaluation of the big picture much
earlier in the design stage than other existing options, simply because it is much
more agnostic regarding hardware details, thus we hope that PET will provide usefull
when developing both tiles for SHMAC and processors in general.

All in all, PET or other tools built from the same concept of weighting architectural
events are indeed possible for a set of scenarios, but it has not yet been validated for
any other architectures. The process of settings weights for PET seems cumbersome, but
for most practical settings the most important thing is to have the weights reasonably
propotioned amongst themselves.
