\chapter{Modifications to gem5}
\label{apx:gem5files}
All file paths are relative to the root of gem5. Diffs are based of gem5-stable revision aaf017eaad7d.

\section{configs/common/Exynos\_4412P.py}
\label{gem5exynos4412p}
\lstinputlisting[language=python,basicstyle=\tiny,numberstyle=\tiny]{../gem5-configs/Exynos_4412P.py}

\section{configs/example/se.py}
\begin{lstlisting}[language=python]
diff --git a/configs/example/se.py b/configs/example/se.py
--- a/configs/example/se.py
+++ b/configs/example/se.py
@@ -104,7 +104,7 @@
         idx += 1
 
     if options.smt:
-        assert(options.cpu_type == "detailed" or options.cpu_type == "inorder")
+        assert(options.cpu_type == "arm_detailed" or options.cpu_type == "inorder")
         return multiprocesses, idx
     else:
         return multiprocesses, 1
@@ -219,7 +219,7 @@
     system.cpu[i].createThreads()
 
 if options.ruby:
-    if not (options.cpu_type == "detailed" or options.cpu_type == "timing"):
+    if not (options.cpu_type == "arm_detailed" or options.cpu_type == "timing"):
         print >> sys.stderr, "Ruby requires TimingSimpleCPU or O3CPU!!"
         sys.exit(1)
 
@@ -255,4 +255,5 @@
     MemConfig.config_mem(options, system)
 
 root = Root(full_system = False, system = system)
+print options
 Simulation.run(options, root, system, FutureClass)
\end{lstlisting}
\vfill

\section{configs/common/CacheConfig.py}
\begin{lstlisting}[language=python]
diff --git a/configs/common/CacheConfig.py b/configs/common/CacheConfig.py
--- a/configs/common/CacheConfig.py
+++ b/configs/common/CacheConfig.py
@@ -55,9 +55,22 @@
 
         dcache_class, icache_class, l2_cache_class = \
             O3_ARM_v7a_DCache, O3_ARM_v7a_ICache, O3_ARM_v7aL2
+    elif options.cpu_type == "exynos_4412p":
+        try:
+            from Exynos_4412P import *
+        except:
+            print "exynos_4412p is unavailable. Did you compile the O3 model?"
+            sys.exit(1)
+
+        dcache_class, icache_class, l2_cache_class = \
+            Exynos_DCache, Exynos_ICache, ExynosL2
+
     else:
         dcache_class, icache_class, l2_cache_class = \
             L1Cache, L1Cache, L2Cache
+    print dcache_class
+    print icache_class
+    print l2_cache_class
 
     # Set the cache line size of the system
     system.cache_line_size = options.cacheline_size
@@ -71,8 +84,9 @@
                                    size=options.l2_size,
                                    assoc=options.l2_assoc)
 
+        print system.cpu_clk_domain
         system.tol2bus = CoherentBus(clk_domain = system.cpu_clk_domain,
-                                     width = 32)
+                                     width = 64)
         system.l2.cpu_side = system.tol2bus.master
         system.l2.mem_side = system.membus.slave
 
\end{lstlisting}
\vfill

\section{configs/common/CpuConfig.py}
\begin{lstlisting}[language=python]
diff --git a/configs/common/CpuConfig.py b/configs/common/CpuConfig.py
--- a/configs/common/CpuConfig.py
+++ b/configs/common/CpuConfig.py
@@ -116,6 +116,13 @@
 except:
     pass
 
+
+try:
+    from Exynos_4412P import Exynos_3
+    _cpu_classes["exynos_4412p"] = Exynos_3
+except:
+    pass
+
 # Add all CPUs in the object hierarchy.
 for name, cls in inspect.getmembers(m5.objects, is_cpu_class):
     _cpu_classes[name] = cls
\end{lstlisting}
\vfill

\section{src/arch/arm/linux/process.cc}
\begin{lstlisting}[language=C++]
diff --git a/src/arch/arm/linux/process.cc b/src/arch/arm/linux/process.cc
--- a/src/arch/arm/linux/process.cc
+++ b/src/arch/arm/linux/process.cc
@@ -66,7 +66,7 @@
 
     strcpy(name->sysname, "Linux");
     strcpy(name->nodename, "m5.eecs.umich.edu");
-    strcpy(name->release, "3.0.0");
+    strcpy(name->release, "3.10.2");
     strcpy(name->version, "#1 Mon Aug 18 11:32:15 EDT 2003");
     strcpy(name->machine, "armv7l");
\end{lstlisting}
