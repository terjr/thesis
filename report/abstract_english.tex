\pagestyle{empty}
\begin{abstract}

\noindent
    * dark silicon \\
    * power wall \\
    * shmac \\
    * good modelling tools \\
    * need for SIMPLE and EARLY STAGE energy estimations \\

    \noindent Since the beginning of semiconductor technology, transistor size has decreased and are still decreasing with a tremendous rate.
This has enabled engineers to build faster and faster single core processors. Faster and more dense processors 

, as there is more space for fancy solutions, and smaller
transistors means shorter switching latencies. In the last couple of years, processors have been kept down by the power wall, which
means that no more power can be dissipated without very clever cooling solutions. This turns out as a lot of space left for transistors
that we cannot use, a phenomenon called dark silicon.






The SHMAC project at NTNU aims to create a heterogeneous computer system that
can make use of dark silicon as space for energy efficient processors and special accelerators. Even though 













\end{abstract}
