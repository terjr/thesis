\section{Test Environment}

PET has already proven that it is able to do the job of power estimation for the
training data. The set of workloads has been split in a training set as
presented in \autoref{sec:workloads} and a test set. This separation is
necessary in order to achieve a good result from the genetic algorithm
\cite{russellnorvig,rajer2003separation}.

The test data set consists of a short Dhrystone run and a synthetic add loop,
each representing very different use of the processor. The Dhrystone test aims
to measure overall performance of the general processor, while the add loop will
stress the ALUs leaving most of the other parts of the CPU idle. The Dhrystone
test is run for $100~000$ iterations. This is too short for a performance
benchmark, but it is long enough to measure current drain and short enough to
simulate on a simple workstation.

There are sources claiming that gem5 is very accurate
\cite{butko2012accuracy,pusdesrissources}, but these claims are done with focus
on overall performance of long-running benchmarks, not the correctness of the
architectural events. This renders a problem for PET, which needs correct
architectural events to happen in order to calculate the power profile. Yet
another important recap is that the gem5 processor model is not identical, but
merely similar to the Exynos~4412. Properties like the fast-loop mode is not
implemented in gem5 and parameters for the branch-predictor is not publicly
available. Further, this means that some discrepancy between the measurements
and PET's prediction is inevitable.
