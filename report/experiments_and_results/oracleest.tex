\section{Power Estimation Challenges}

Creating PET has revealed a number of challenges of simulator supported power
estimation. The goal of any power estimation tool would be to deliver the
correct answer of how much power a certain architecture would use, but many
caveats makes this almost impossible with current methods. A non-complete list
of things that are important with respect to power consumption, but are hard to
get correct in a simulator are described below.

\begin{description}
    \item[Cache and memory state:]
        The cache and memory systems, together with various cache prefetchers,
        branch predictors, fast-loop queues and so on will make it difficult to
        assure that the simulated system are in the exact same state as the
        physical system.
    \item[Interrupts:]
        It is hard to predict when interrupts occur. Interrupts causes context
        switches, and thus a change in system behavior and power drain.
    \item[Undisclosed CPU and system specifications:]
        Unless the whole system is built in-house, there will often be certain
        specifications that are not available to the tester. PET was built
        against a CPU where many details were undisclosed; it was not trivial to
        find sufficient details to configure the simulator.
\end{description}

The above list is by no means a show stopper. However, half-measures must be
taken if a non-complete simulator is used. Nevertheless, fruitful results can
still be obtained. The best power estimator that could be built around a
simulator would be the one that reflected the ideas of the simulated hardware in
the best possible way. This estimator would give a hint about final power
consumption and its trend over a set of test programs, and would be usable for
application specific power optimization.
